%!TEX root = ../main.tex
\chapter{Annexe I: Les pages annexes}
Les pages annexes apportent un complément d’information au corps du travail et ne doivent contenir que ce qui est important pour la compréhension de la thèse ou ce qui en supporte l’argumentation. Il est important de ne pas inclure n’importe quoi dans les annexes simplement pour sauver du temps, en se rappelant qu’elles seront publiées sur internet.

Les annexes sont placées après la bibliographie. S’il y en a plusieurs, elles sont présentées selon l’ordre de mention dans le texte avec la numérotation Annexe I, Annexe II, Annexe III, etc. Numérotation en chiffres romains majuscules suivie de leur propre titre.

D’autres listes peuvent être ajoutées au besoin. Par exemple, liste des publications ou documents inclus dans le texte ou hors texte, d’abréviations, sigles ou symboles, d’annexes, etc.

\section{Lorem ipsum}
Lorem ipsum dolor sit amet, consectetur adipiscing elit. Sed tortor quam, facilisis in magna at, convallis fringilla risus. Curabitur adipiscing imperdiet ligula ut congue. Nunc non urna sed velit aliquet tempor. Fusce tempus vestibulum pharetra. Sed laoreet erat sed odio cursus, et egestas lacus fringilla. Nullam sollicitudin condimentum scelerisque. Morbi fermentum blandit purus, eget feugiat quam egestas at. Nam at velit ac velit feugiat lacinia. Mauris quis tortor dignissim, feugiat justo eget, interdum tortor. Proin vel lacus non eros eleifend aliquet nec at elit. Integer rhoncus laoreet ligula.
